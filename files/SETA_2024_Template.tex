\documentclass[12pt,twoside]{article}


%%%%%%%%%%%%%%%%%%%%%%%%%%%%%%%%%%%%%%%%%%

\usepackage[a4paper]{geometry}
\setlength{\textwidth}{6.3in}
\setlength{\textheight}{8.8in}
\setlength{\topmargin}{0pt}
\setlength{\headsep}{25pt}
\setlength{\headheight}{0pt}
\setlength{\oddsidemargin}{0pt}
\setlength{\evensidemargin}{0pt}

%%%%%%%%%%%%%%%%%%%%%%%%%%%%%%%%%%%%%%%%%%

\makeatletter
\renewcommand\title[1]{\gdef\@title{\reset@font\Large\bfseries #1}}
\renewcommand\section{\@startsection {section}{1}{\z@}%
                                   {-3.5ex \@plus -1ex \@minus -.2ex}%
                                   {2.3ex \@plus.2ex}%
                                   {\normalfont\large\bfseries}}
\renewcommand\subsection{\@startsection{subsection}{2}{\z@}%
                                     {-3ex\@plus -1ex \@minus -.2ex}%
                                     {1.5ex \@plus .2ex}%
                                     {\normalfont\normalsize\bfseries}}
\renewcommand\subsubsection{\@startsection{subsubsection}{3}{\z@}%
                                     {-2.5ex\@plus -1ex \@minus -.2ex}%
                                     {1.5ex \@plus .2ex}%
                                     {\normalfont\normalsize\bfseries}}

\def\@runningauthor{}\newcommand{\runningauthor}[1]{\def\runningauthor{#1}}
\def\@runningtitle{}\newcommand{\runningtitle}[1]{\def\runningtitle{#1}}

\renewcommand{\ps@plain}{%
\renewcommand{\@evenfoot}{\footnotesize Sequences and Their Applications (SETA) 2024\hfill\thepage}
\renewcommand{\@oddfoot}{\footnotesize Sequences and Their Applications (SETA) 2024 \hfill\thepage}
\renewcommand{\@evenhead}{\footnotesize\scshape \hfill\runningauthor\hfill}
\renewcommand{\@oddhead}{\footnotesize\scshape \hfill\runningtitle\hfill}}
\pagestyle{plain}

\g@addto@macro\bfseries{\boldmath}

\makeatother

%%%%%%%%%%%%%%%%%%%%%%%%%%%%%%%%%%%%%%%%%%%%%%%%%%%%%%%%%%%%%%%%%%%%%%%%

% Please remove all other commands that change parameters such as
% margins or pagesizes.

% we recommend these ams packages
\usepackage{amsthm,amsmath,amssymb}

% we recommend the graphicx package for importing figures
\usepackage{graphicx}

% hyperlinks
\usepackage[colorlinks=true,citecolor=black,linkcolor=black,urlcolor=blue]{hyperref}

% declare theorem-like environments
\theoremstyle{plain}
\newtheorem{theorem}{Theorem}
\newtheorem{lemma}[theorem]{Lemma}
\newtheorem{corollary}[theorem]{Corollary}
\newtheorem{proposition}[theorem]{Proposition}
\newtheorem{fact}[theorem]{Fact}
\newtheorem{observation}[theorem]{Observation}
\newtheorem{claim}[theorem]{Claim}

\theoremstyle{definition}
\newtheorem{definition}[theorem]{Definition}
\newtheorem{example}[theorem]{Example}
\newtheorem{conjecture}[theorem]{Conjecture}
\newtheorem{open}[theorem]{Open Problem}
\newtheorem{problem}[theorem]{Problem}
\newtheorem{question}[theorem]{Question}

\theoremstyle{remark}
\newtheorem{remark}[theorem]{Remark}
\newtheorem{note}[theorem]{Note}

%%%%%%%%%%%%%%%%%%%%%%%%%%%%%%%%%%%%%%%%%%%%%%%%%%%%%%%

%% Enter a title

\title{A proof of a conjecture on\\sequences with unusual properties}

%% Enter a running title

\runningtitle{A proof of a conjecture on sequences with unusual properties}

%% Enter authors and affiliations

\author{First Author\thanks{First Author is supported by funding organisation xyz.}\\
\small Department of Mathematics\\[-0.8ex]
\small Southpole University\\[-0.8ex] 
\small Southpole, Antarctica\\
\small\tt author@math.southpole.edu\\
\and
Possible Second Author \qquad  Possible Third Author\\
\small School of Computer Science\\[-0.8ex]
\small University of Nowhere\\[-0.8ex]
\small Nowhere, Forgottencountry\\
\small\tt $\{$psa,pta$\}$@uni.fc
}

%% Enter running author names

\runningauthor{F.\ Author, P.\ S.\ Author, P.\ T.\ Author}

\date{}


\begin{document}


\maketitle

\thispagestyle{empty}

\begin{abstract}
The paper should begin with a clear and informative abstract.
\end{abstract}

%%%%%%%%%%%%%%%%%%%%%%%%%%%%%%%%%%%%%%%%%%%%%%%%%%%%%%%

\section{Introduction}

We prove a conjecture due to John Smith~\cite{Smith1950} concerning sequences with unusual properties.

\begin{theorem}
\label{thm:seq}
Sequences with unusual properties exist.
\end{theorem}

We shall prove Theorem~\ref{thm:seq} using a new method, which we call the magical method.

%%%%%%%%%%%%%%%%%%%%%%%%%%%%%%%%%%%%%%%%%%%%%%%%%%%%%%%

\section{The magical method}

In this section we describe our main method.


\begin{lemma}
\label{lem:technical}
If a sequence satisfies property A, then it satisfies property B.
\end{lemma}
\begin{proof}
Suppose for a contradiction that a sequences satifies property A, but does not property B. \dots
\end{proof}

%%%%%%%%%%%%%%%%%%%%%%%%%%%%%%%%%%%%%%%%%%%%%%%%%%%%%%%

\section{Proof of Theorem~\ref{thm:seq}}

In this section we complete the proof of Theorem~\ref{thm:seq}.

\begin{proof}[Proof of Theorem~\ref{thm:seq}]
Argument \dots. This completes the proof of Theorem~\ref{thm:seq}.
\end{proof}

%% If you want to use figures, use the following.

% \begin{figure}[!h]
%   \begin{center}
%     % use \includegraphics to import figures 
%     % \includegraphics{filename}
%   \end{center}
%   \caption{\label{fig:InformativeFigure} Here is an informative
%     figure.}
% \end{figure}


%%%%%%%%%%%%%%%%%%%%%%%%%%%%%%%%%%%%%%%%%%%%%%%%%%%%%%%

\subsection*{Acknowledgements}

Possible acknowledgements.

%%%%%%%%%%%%%%%%%%%%%%%%%%%%%%%%%%%%%%%%%%%%%%%%%%%%%%%

% \bibliographystyle{plain} 
% \bibliography{ref} 
% Use BibTeX to create a bibliography
% then copy and past the contents of your .bbl file into your .tex file

\begin{thebibliography}{1}

\bibitem{Smith1950}
J.~Smith.
\newblock A conjecture on sequences with unusual properties.
\newblock {\em J. Major Results}, 5(2):100--200, 1950.

\end{thebibliography}

\end{document}
